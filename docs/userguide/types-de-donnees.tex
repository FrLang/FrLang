\documentclass[../userguide.tex]{subfiles}

\begin{document}
    \parindent
    En \lang, les données sont typées fortement et statiquement, c'est-à-dire que chaque variable est associée à un
    type de valeur, et qu'une fois que cette variable a pris un type, celle-ci ne peut plus en changer.

    Les types de valeurs disponibles en \lang{} sont : \\

    \begin{tabularx}{\columnwidth}{| l | X | l |}
        \hline
        \textbf{Type}     & \textbf{Description}                                       & \textbf{Exemple}  \\ \hline
        \dcode{entier}    & Représente un nombre entier qu'il soit positif ou négatif  & \dcode{5}         \\ \hline
        \dcode{decimal}   & Représente un nombre décimal qu'il soit positif ou négatif & \dcode{3.2}       \\ \hline
        \dcode{booleen}   & Représente un booléen (vrai ou faux)                       & \dcode{vrai}      \\ \hline
        \dcode{texte}     & Représente du texte                                        & \dcode{"bonjour"} \\ \hline
        \dcode{caractere} & Représente un caractère                                    & \dcode{'a'}       \\ \hline
        \dcode{?}         & Représente un type inconnu                                 & \dcode{?}         \\ \hline
    \end{tabularx}
    \parindent \\

    \subsection{Types absolus} \label{subsec:types-absolus}
    \parindent
    Les types \code{entier} et \code{decimal} peuvent perdre leur capacité à devenir négatif s'ils sont suivis
    du mot clé \code{absolu}, créant ainsi les types \code{entier absolu} et \code{decimal absolu}.

    \subsection{Préciser un type de données à une variable} \label{subsec:preciser-un-type-de-donnees-a-une-variable}
    \parindent
    Il est possible, au moment de la création d'une variable, de préciser son type.
    Cela permet de déjà connaître, et avant même son initialisation, le type de la variable.
    En cas d'initialisation au moment de la déclaration de la variable, le type de la variable est déduit à partir
    de sa valeur.
    Il n'est donc jamais utile de préciser le type d'une constante. \\
    La syntaxe pour préciser le type d'une variable est : \code{var nom: type}.

    \begin{lstlisting}[label={lst:var-type-ex-1}]
        var a: entier; // Creation d'une variable "a" de type entier
        var b: decimal absolu, c; // Creation d'une variable "b" de type decimal absolu, et d'une variable "c" de type inconnu
    \end{lstlisting}

    \subsection{Type inconnu et pointeur nul} \label{subsec:type-inconnu-et-pointeur-nul}
    \parindent
    Le type \code{?} est un type inconnu, il est utilisé pour désigner un type qui n'est pas encore défini, par
    exemple le type d'une variable non assignée, et sans type précisé.
    C'est le seul type de variable qui peut changer après sa déclaration.

    \begin{lstlisting}[label={lst:var-type-ex-2}]
        var a: ?; // Creation d'une variable "a" de type inconnu
        a = 5; // On assigne la valeur 5 a la variable "a". Le type de la variable "a" est maintenant entier.
    \end{lstlisting}

    Le pointeur nul quant à lui est une valeur qui peut être donnée à une variable de n'importe quel type et qui
    permet de dire que la variable n'a pas de valeur.
    C'est la valeur qui est utilisée lorsque la variable n'a pas encore été initialisée.
    Le pointeur nul est donc la seule instance du type \code{?}.
\end{document}
