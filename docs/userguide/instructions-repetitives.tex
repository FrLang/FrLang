\documentclass[../userguide.tex]{subfiles}

\begin{document}

    Les instructions répétitives, ou plus couramment appelées boucles sont des instructions destinées à être exécutées
    plusieurs fois.\\
    Il existe quatre types principaux de boucles :
    \begin{itemize}
        \item \code{tant que} : Permet de répéter une instruction tant que la condition est vraie.
        \item \code{faire tant que} : Permet de répéter une instruction une fois ou plus tant que la condition est
        vraie.
        \item \code{pour} : Permet de créer une boucle qui s'exécute d'une certaine quantité de fois.
        \item \code{pour chaque} : Permet de parcourir un tableau ou une liste et d'exécuter une instruction à chaque
        itération.
    \end{itemize}

    \subsection{Les boucles \dcode{tant que} et \dcode{faire tant que}} \label{subsec:boucle-tant-que}
    \parindent
    \begin{lstlisting}[label=lst:boucle-tant-que-syntaxe]
        tant que condition
            instruction | bloc
    \end{lstlisting}

    \begin{lstlisting}[label=lst:boucle-faire-tant-que-syntaxe]
        faire
            instruction | bloc
        tant que condition;
    \end{lstlisting}
    \divider

    La boucle \code{tant que} permet d'exécuter une série d'instructions tant que la condition qui lui est
    donnée est vraie. \\
    Elle vérifie avant chaque itération si la condition est vraie et exécute les instructions contenues dans la boucle
    si c'est le cas.

    \begin{lstlisting}[label=lst:boucle-ex-1]
        var a = "";
        tant que taille(a) < 10 {
            a += "a";
        }
        .a;
    \end{lstlisting}

    Cette boucle exécute l'instruction \code{a += "a"} tant que la longueur de tu texte \code{a} est inférieure à 10,
    et affiche le contenu de \code{a} à la fin.

    La boucle \code{faire tant que} permet elle aussi d'exécuter une série d'instructions tant que la condition qui lui
    est donnée est vraie.
    Cependant, elle vérifie si la condition est vraie après chaque itération, permettant donc de s'assurer que le
    contenu de la boucle est exécuté au moins une fois.

    \begin{lstlisting}[label=lst:boucle-ex-2]
        var a: texte;
        faire {
            a = "a";
        } tant que faux;
        .a;
    \end{lstlisting}
    \begin{lstlisting}[label=lst:boucle-ex-3]
        var a: texte;
        tant que faux {
            a = "a";
        }
        .a;
    \end{lstlisting}

    Dans le \hyperref[lst:boucle-ex-2]{premier cas}, l'instruction \code{.a;} à la fin affichera \code{a}, alors que dans
    le \hyperref[lst:boucle-ex-3]{second cas}, cette même instruction affichera \code{nul}.

    \subsection{Les boucles pour}\label{subsec:les-boucles-pour}
    \parindent
    Il existe trois types de boucles pour :
    \begin{itemize}
        \item \code{pour} : Permet de créer une boucle avec un compteur qui s'incrémente d'une certaine quantité un
        certain nombre de fois.
        \item \code{pour var} : Permet de créer une boucle \code{pour} avec un comportement personnalisé.
        \item \code{pour chaque} : Permet de parcourir un tableau ou une liste et d'exécuter une instruction à chaque
        itération.
    \end{itemize}

    \subsubsection{La boucle \dcode{pour}}
    \parindent
    \begin{lstlisting}[label=lst:boucle-pour-syntaxe]
        pour nom = début -> fin [(pas)]
            instruction | bloc
    \end{lstlisting}
    \divider

    Cette boucle permet de créer une variable qui ira d'un nombre à un autre, avec un certain interval.
    L'interval est optionnel et vaut 1 par défaut.
    Si le nom de la variable donné existe déjà, la variable existante sera écrasée, et la valeur de la variable après
    l'itération sera la valeur d'arrivée de la boucle.

    \begin{lstlisting}[label=lst:boucle-pour-1]
        pour i = 1 -> 10 {
            .i;
        }
    \end{lstlisting}

    \begin{lstlisting}[label=lst:boucle-pour-2]
        pour i = 1 -> 10 (2) {
            .i;
        }
    \end{lstlisting}

    Le \hyperref[lst:boucle-pour-1]{premier exemple} affiche les entiers de 1 à 10, et le
    \hyperref[lst:boucle-pour-2]{second exemple} affiche les entiers de 1 à 10 de 2 en 2.

    \subsubsection{La boucle \dcode{pour var}}
    \parindent
    \begin{lstlisting}[label=lst:boucle-pour-var-syntaxe]
        pour [initialisation]; [condition]; [etape]
            instruction | bloc
    \end{lstlisting}
    \divider

    La boucle \code{pour var} permet de créer un variable qui sera modifiée et vérifiée à chaque itération.
    Cette boucle est séparée en trois parties, séparées par un point-virgule :
    \begin{itemize}
        \item \code{initialisation} : Création de la variable unique à la boucle.
        \item \code{condition} : Condition vérifiée avant chaque itération.
        \item \code{etape} : Modification apportée à la variable après chaque itération.
    \end{itemize}

    \begin{lstlisting}[label=lst:boucle-pour-var-1]
        pour var i = 0; i <= 5; i += 1 {
            .i;
        }
    \end{lstlisting}

    Cette boucle est équivalente à la boucle \code{pour} suivante :
    \begin{lstlisting}[label=lst:boucle-pour-var-2]
        pour i = 0 -> 5 {
            .i;
        }
    \end{lstlisting}

    \subsubsection{La boucle \dcode{pour chaque}}
    \parindent
    \begin{lstlisting}[label=lst:boucle-pour-chaque-syntaxe]
        pour chaque [nom] dans [tableau]
            instruction | bloc
    \end{lstlisting}
    \divider

    La boucle \code{pour chaque} permet de parcourir un tableau ou une liste et d'exécuter une action à chaque
    itération.

    \begin{lstlisting}[label=lst:boucle-pour-chaque-1]
        const eleves = ["Tom", "Laura", "Bob", "John"];
        pour chaque eleve dans eleves {
            .eleve;
        }
    \end{lstlisting}

\end{document}
