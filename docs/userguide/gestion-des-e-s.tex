\documentclass[../userguide.tex]{subfiles}

\begin{document}
    Les entrées \& sorties sont une partie importante d'un langage de programmation.
    En effet, c'est un moyen de communiquer entre le programmeur et l'utilisateur.
    Il est donc important que la syntaxe des entrées et sorties soit minimale, afin de ne pas perdre de temps et de
    place inutilement.
    C'est pourquoi, dans le langage \lang, les entrées et sorties sont définies par une syntaxe simple et concise :

    \subsection{Sorties} \label{subsec:sorties}
    \parindent
    Les sorties sont définies par le symbole \code{.} suivi de l'expression à afficher.

    \begin{lstlisting}[label={lst:sorties-ex-1}]
        . "Bonjour"; // Affichage du texte "Bonjour"
        .a; // Affichage de la valeur de la variable "a"
        .5; // Affichage de valeur "5"
        .0.4; // Affichage de "0.4"
    \end{lstlisting}

    \subsection{Entrées} \label{subsec:entrees}
    \parindent
    Pour demander une valeur à l'utilisateur, il suffit d'écrire \code{???} dans le code.
    Cela demandera alors à l'utilisateur de saisir une valeur.
    Les variables résultant de cette saisie sont par défaut des variables de type \code{texte}, mais il est possible
    d'automatiquement les transformer en variables d'un autre type primitif, en spécifiant explicitement leur type.

    \begin{lstlisting}[label={lst:entrees-ex-1}]
        var a = ???; // Saisie de la valeur de la variable "a" (texte)
        var b: entier = ???; // Saisie de la valeur de la variable "b" (entier)
    \end{lstlisting}
\end{document}
