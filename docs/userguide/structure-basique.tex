\documentclass[../userguide.tex]{subfiles}

\begin{document}

    \subsection{Structure d'une ligne} \label{subsec:structure-d'une-ligne}
    \parindent
    Toutes les instructions de \lang{} peuvent être séparées sur plusieurs lignes sous réserve qu'aucun mot-clé ou
    identifiant ne soit coupé par ce saut de ligne.
    Elles doivent toutes être séparées par un \code{;} (habituellement en fin de ligne).

    \subsection{Structure d'un bloc} \label{subsec:structure-d'un-bloc}
    \parindent
    Les blocs sont une structure de code contenant des instructions.
    Ils commencent par \code{\{} et se terminent par \code{\}}.
    Toutes les instructions d'un block doivent être indentées d'un niveau au dessus du niveau précédent.
    Ils ont la particularité de créer une portée de variables qui n'est accessible que dans le bloc courant,
    et dans les éventuels blocs sous-jacent.

\end{document}
