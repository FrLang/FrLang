\documentclass[../userguide.tex]{subfiles}

\begin{document}
    \begin{lstlisting}[label=lst:conditions-syntaxe]
        si condition
            instruction | bloc
        [sinon
            instruction | bloc]
    \end{lstlisting}
    \divider

    Les instructions conditionnelles sont une grande partie des langages de programmation impératifs.
    Elles permettent de définir des conditions dans lesquelles un bloc de code sera exécuté.
    La syntaxe de ces instructions est la suivante :
    \begin{lstlisting}[label=lst:conditions-ex-1]
        si condition1 {
            // code a exécuter si la condition1 est vraie
        } sinon si condition2 {
            // code a exécuter si la condition1 est fausse, mais si la condition2 est vraie
        } sinon {
            // code a exécuter si aucune condition n'est vraie
        }
    \end{lstlisting}

\end{document}
